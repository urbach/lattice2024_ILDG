% Please make sure you insert your data according to the instructions in
%  PoSauthmanual.pdf
\documentclass[a4paper,11pt]{article} \usepackage{pos} \usepackage{subcaption}

% In this document,
% we use semantic line breaks
% (https://sembr.org)

\title{ILDG 2.0}

\author[a]{Hideo Matsufuru}
\affiliation[a]{XX}
\emailAdd{XX}


\author[b]{Hubert Simma}
\affiliation[b]{John von Neumann-Institut für Computing NIC, Deutsches Elektronen-Synchrotron DESY,
Platanenallee 6, 15738 Zeuthen, Germany}
\emailAdd{hubert.simma@desy.de}

\ExplSyntaxOn
\seq_put_right:Nn \l_authors_for_head_s{Carsten\ Urbach\ et\ al.}
\ExplSyntaxOff


\author[c]{Carsten Urbach}
\affiliation[c]{Helmholtz-Institut f\"ur Strahlen- und Kernphysik (Theorie) and Bethe, Center for Theoretical Physics, Universit\"at Bonn, 53115 Bonn, Germany}
\emailAdd{urbach@hiskp.uni-bonn.de}



\abstract{
}

\FullConference{The 41st International Symposium on Lattice Field Theory (LATTICE2024)\\
 28 July - 3 August 2024\\
Liverpool, UK\\}


\begin{document}
\maketitle


\section{Introduction}

The International Lattice Data Grid (ILDG) represents an joint effort
of the Lattice QCD (LQCD) community, which started some 20 years ago
with the goal to share valuable gauge configurations worldwide. The
ILDG was implemented as a federation of autonomous regional grids,
CSSM for Australia, JLDG for Japan, Latfor DataGrid (LDG) for
continental Europe, UK Lattice Field Theory for the UK, and USQCD for
the US.

Each regional grid autonomously operates the ILDG core services: a
website, a metadata catalogue (MDC), a file catalogue (FC), and
storage elements (SM). The ILDG itself forms a virtual organisation
providing in particular a user registration and authentication
service, which implement the international interoperability of the
regional grids. The user registration and authentication service
originally used the virtual organisation membership service (VOMS),
which is now replaced by an identity and access management (IAM)
service, see below. 

The ILDG as an organisation is borne by the ILDG board, the Metadata
Working Group (MDWG), and the Middleware Working Group (MWG). The ILDG
board is responsible for administrative and organisational matters of
the ILDG. The MDWG is responsible for the specification and update of
metadata schemata and file formats, and the MWG for the underlying
services. While not explicitly intendet at the time, the ILDG
implements the FAIR principles, i.e Findability, Accessibility,
Interoperability, and Reusability.

Due to lack of resources, the ILDG services unfortunately degraded
over the last decade. Only due to funding obtained as part of the NFDI
(national research data initiative) funding scheme of the German
Research Foundation (DFG) and the NFDI funded PUNCH4NFDI (Paricles,
Universe, Nuclei and Hadrons for the NFDI) consortium it became
possible to refurbish, modernise, and resume the ILDG services. In
parallel, a significant effort was made to extend the metadata
schemata and file formats to serve modern requirements of the LQCD
community and beyond.

These efforts led to a modern re-implementation of the metadata and
file catalogue services based on the REST API allowing for a simple
containerised deployment, a new user registration and authentication
service based on IAM hosted at CNAF (INFN) based on a memorandum of
understanding between CNAF and ILDG, new versions of the configuration
and ensemble metadata schemata as well as a new version of the file
format, and prototype implementations of client tools for the command
line, with a GUI, and with a web interface. Details will be discussed
in the reminder of this proceeding contribution.

\section{Metadata Schemata}

Metadata schemata aim to provide means to markup data in a
standardised and unique way. At the same time, such schemata should be
easily extensible and flexible to handle. The Extensible Markup
Language (XML) fulfils these requirements: it is a W3C standard, it
provides powerful query and search technology via XPATH, and well
tested and open source software implementations are
available. Moreover, with the XML Schema Definition (XSD), schemata
and unique data markup can be standardised. While mainly aimed
for machine readability, XML is also half-decently human readable.
For these reasons the ILDG schemata are based on XML/XSD, which worked
very successfully since the beginning.

Nowadays, JSON or YAML could be used as alternatives to XML. However, for
backward compatibility, ILDG will stick to XML/XSD. Also, there are no
obvious advantages of JSON or YAML over XML, apart from YAML being
slightly easier to read for humans, and JSON and/or YAML description
can be in principle always converted into XML (up to some caveats, we
are not going to discuss here).

The structure of the markup in ILDG is as follows: the binary (gauge
configuration) data is 
stored as one so-called message in a container file (lime format)
together with another message containing a minimal markup of this
data, including a checksum and its site-independent logical file name
(LFN). This binary file needs to follow the \emph{ILDG binary format}.
This binary data is stored on a storage element and marked up
by a separate \emph{configuration XML file}. This configuration XML file is
required to validate against the ILDG configuration XML schema. Apart
from some configuration specific information like for instance the
plaquette value, the configuration XML file links to the binary file
via the LFN, but also to the Markov chain or ensemble the
configuration belongs to via the Markov Chain Uniform Resource
Identifier (markovChainURI). Finally, the ensemble is marked up using
an \emph{ensemble XML file}, which need to validate against the ILDG
ensemble XML schema. This XML file compiles all information common to all
configurations that are part of the ensemble. This information
includes for instance the algorithm description and the lattice action.

For the entension of the file format and ILDG XML schemata, the
community was queried, and the MDWG reviewed, discussed and
implemented most of the requests. The changes are compiled in an
extensive document \emph{cite!?}, with an overview provided in the
following.

Common changes to both, the configuration and the ensemble XML schema
concern the possibility to specify the ORCID as an alternative to
specifying name and institution in the \texttt{<participant>}
element. Thus, the participant can be describe by an \texttt{<orcid>}
element, followed by optional \texttt{<name>} and/or
\texttt{<institution>}, or the by latter two.

Moreover, at top-level there is a new
\texttt{<additionalInfo>} free (and optional) sub-tree, which can be
used to provide additional information. Several other elements allow
now for a new \texttt{<annotation>} sub-element in order to be able to
provide more specific additional information. The new
\texttt{<annotation>} element also replaces the former
\texttt{<comment>} element.

\subsection{Ensemble XML}

The ensemble XML schema was extended by three major changes. First,
there is now the possibility to specify a funding reference via one or
more \texttt{<fundingReference>} elements as a non-empty list of
\texttt{<fundingReference>} elements. The MDWG decided to follow
here the example of the DataCite schema, however, restricted to a
subset of it. Every funding reference has a mandatory
\texttt{<funderName>} element and/or optional \texttt{<awardTitle>}
and \texttt{<awardNumber>} child elements.

Second, a license must be specified as part of the ensemble XML, which
is a requirement to be compliant with the FAIR principles. One can
specify standard licensis, like e.g. a GNU public license, or a custom
license. This \texttt{<license>} element comes with two important
additional possibilities: one is the possibility to specify an embargo
data, the other is to provide a way to specify how the usage of the
ensemble must be acknowleged in a respective publication.
For more details we refer to the MDWG document.

Third, the list of lattice actions and gauge groups has been
significantly extended to reflect the progress in the field. The
supported gauge groups are now SU$(N)$ and SO$(N)$ for $N\geq 2$,
SP$(N)$ with even $N\geq 4$ and U$(N)$ with $N\geq1$.

On the lattice action side, several new action types have been
defined. To account for the growing number of works including QED
effects, QED gauge actions are available as new XML elements.
Higher fermion representations are now
possible to markup, including for instance fermions in the adjoint
represenation. Open boundary conditions with and without Schrödinger
functional can be marked up, Möbius domain-wall fermions as well as
Wilson clover fermions with exponetial clover term.

Finally, it is possible to specify fermions coupled to electromagnetic
fields. Initially, the ensemble XML schema includes so-called SLINC
fermions as used by QCDSF and the coupling in the framework of
$C^\star$ boundary conditions.

\textbf{do we want a table with the new action specifiers?}

\subsection{Configuration XML}

The main change in the ILDG configuration XML schema is due to the
request to store multiple gauge configurations in a single binary
file. In order to make this possible, the one-to-one correspondence
between one single gauge configuration binary data and one single
configuration XML file had to abandoned. Instead of a single
\texttt{<markovStep>} element, there is 
now a single \texttt{<markovSequence>} element per configuration XML
file. The \texttt{<markovSequence>} can have one or more
\texttt{<markovStep>} elements as child elements. The latter have the
checksum stored in \texttt{<crcCheckSum>} and the plaquette in
\texttt{<avePlaquette>} sub-elements. For consistency, the Markov
Chain URI (\texttt{<markovChainURI>}) and the \texttt{<series>}
elements are now mandatory sub-element of \texttt{<markovSequence>}. 

In the context of these changes, also the LFN was moved to become the
first top-level element (\texttt{<dataLFN>}) in the configuration schema.

\subsection{ILDG file format}

\section{ILDG Core Services}

\subsection{Metadata and File Catalogues}

\subsection{User Registration and Authentication}

One might be wondering why user registration and authetication is
required, in particular in the context of the open science movement
(cite the panel discussion). However, we are convinced that even with
open science in mind authentication and, thus, user registration is a
fundamental requirement.

First of all, the data producers in the LQCD community expressed the
need for so-called embargo time, which are a way guarantee first usage
rights to the data producers. The main motivation for this requirement
is the very fact that the production of gauge configuration ensembles
represents a significant effort, which is however usually not
easily publishable alone. Therefore, in order to avoid double storage,
there must be mechanisms for access control. Data usage also comes
with certain rules, which can only be implemented based on personal
access rights.

Second, the storage providers have to provide in addition to the
storage space also network bandwidth enabling the access to the
data. This infrastructure is expensive and must not be saturated with
arbitrary downloads. Moreover, not everyone can be allowed to upload
data to the storage elements. For this reason also storage providers
usually require access control mechanisms.

\section{Conclusions}


% TODO: Does this make sense to include?
We would encourage those interested in driving such work forward
to contact us,
and to join relevant mailing lists of the ILDG~\cite{ildg-organization},
% TODO is this true?
where there may be scope to start one or more new working groups
focusing specifically on the topics discussed here.


\acknowledgments

H.M. \dots

H.S. \dots

The work of C.U. on this project was suppoerted in parts by the
Deutsche Forschungsgemeinschaft (DFG, German Research Foundation) as
part of the CRC 1639 NuMeriQS – project no. 511713970. 


\paragraph*{Open access statement}
For the purpose of open access, the authors have applied a Creative Commons
Attribution (CC BY) licence to any Author Accepted Manuscript version arising.

\paragraph*{Research Data Access Statement}
No new data were generated during the preparation of this work.

\bibliography{references}
\bibliographystyle{apsrev}
\end{document}
