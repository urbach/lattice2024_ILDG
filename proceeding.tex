% Please make sure you insert your data according to the instructions in
%  PoSauthmanual.pdf
\documentclass[a4paper,11pt]{article} \usepackage{pos} \usepackage{subcaption}

% In this document,
% we use semantic line breaks
% (https://sembr.org)

\title{ILDG 2.0}

\author[a]{Hideo Matsufuru}
\affiliation[a]{XX}
\emailAdd{XX}


\author[b]{Hubert Simma}
\affiliation[b]{John von Neumann-Institut für Computing NIC, Deutsches Elektronen-Synchrotron DESY,
Platanenallee 6, 15738 Zeuthen, Germany}
\emailAdd{hubert.simma@desy.de}

\ExplSyntaxOn
\seq_put_right:Nn \l_authors_for_head_s{Carsten\ Urbach\ et\ al.}
\ExplSyntaxOff


\author[c]{Carsten Urbach}
\affiliation[c]{Helmholtz-Institut f\"ur Strahlen- und Kernphysik (Theorie) and Bethe, Center for Theoretical Physics, Universit\"at Bonn, 53115 Bonn, Germany}
\emailAdd{urbach@hiskp.uni-bonn.de}



\abstract{
}

\FullConference{The 41st International Symposium on Lattice Field Theory (LATTICE2024)\\
 28 July - 3 August 2024\\
Liverpool, UK\\}


\begin{document}
\maketitle


\section{Introduction}

The International Lattice Data Grid (ILDG) represents an joint effort
of the Lattice QCD (LQCD) community, which started some 20 years ago
with the goal to share valuable gauge configurations worldwide. The
ILDG was implemented as a federation of autonomous regional grids,
CSSM for Australia, JLDG for Japan, Latfor DataGrid (LDG) for
continental Europe, UK Lattice Field Theory for the UK, and USQCD for
the US.

Each regional grid autonomously operates the ILDG core services: a
website, a metadata catalogue (MDC), a file catalogue (FC), and
storage elements (SM). The ILDG itself forms a virtual organisation
providing in particular a user registration and authentication
service, which implement the international interoperability of the
regional grids. The user registration and authentication service
originally used the virtual organisation membership service (VOMS),
which is now replaced by an identity and access management (IAM)
service, see below. 

The ILDG as an organisation is borne by the ILDG board, the Metadata
Working Group (MDWG), and the Middleware Working Group (MWG). The ILDG
board is responsible for administrative and organisational matters of
the ILDG. The MDWG is responsible for the specification and update of
metadata schemata and file formats, and the MWG for the underlying
services. While not explicitly intendet at the time, the ILDG
implements the FAIR principles, i.e Findability, Accessibility,
Interoperability, and Reusability.

Due to lack of resources, the ILDG services unfortunately degraded
over the last decade. Only due to funding obtained as part of the NFDI
(national research data initiative) funding scheme of the German
Research Foundation (DFG) and the NFDI funded PUNCH4NFDI (Paricles,
Universe, Nuclei and Hadrons for the NFDI) consortium it became
possible to refurbish, modernise, and resume the ILDG services. In
parallel, a significant effort was made to extend the metadata
schemata and file formats to serve modern requirements of the LQCD
community and beyond.

These efforts led to a modern re-implementation of the metadata and
file catalogue services based on the REST API allowing for a simple
containerised deployment, a new user registration and authentication
service based on IAM hosted at CNAF (INFN) based on a memorandum of
understanding between CNAF and ILDG, new versions of the configuration
and ensemble metadata schemata as well as a new version of the file
format, and prototype implementations of client tools for the command
line, with a GUI, and with a web interface. Details will be discussed
in the reminder of this proceeding contribution.



\section{Conclusions}


% TODO: Does this make sense to include?
We would encourage those interested in driving such work forward
to contact us,
and to join relevant mailing lists of the ILDG~\cite{ildg-organization},
% TODO is this true?
where there may be scope to start one or more new working groups
focusing specifically on the topics discussed here.


\acknowledgments

H.M. \dots

H.S. \dots

The work of C.U. on this project was suppoerted in parts by the
Deutsche Forschungsgemeinschaft (DFG, German Research Foundation) as
part of the CRC 1639 NuMeriQS – project no. 511713970. 


\paragraph*{Open access statement}
For the purpose of open access, the authors have applied a Creative Commons
Attribution (CC BY) licence to any Author Accepted Manuscript version arising.

\paragraph*{Research Data Access Statement}
No new data were generated during the preparation of this work.

\bibliography{references}
\bibliographystyle{apsrev}
\end{document}
